% !TeX root = main.tex
% !TEX program = pdflatex

\chapter{Elementos matemáticos y tipográficos}

\section{Ecuaciones en el estilo Tufte}

Las ecuaciones pueden colocarse en línea como \(E = mc^2\) o en modo display:

\begin{equation}
\int_a^b f(x)\,dx = F(b) - F(a)
\end{equation}

\subsection{Ejemplo de figura completa}

\begin{figure*}[ht]\centering
%\includegraphics[width=\textwidth]{imagenes/diagrama-completo.png}
% Imagen no encontrada. Actualiza la ruta o agrega el archivo correspondiente.
\caption{Diagrama que ocupa el ancho completo de la página, incluyendo el margen. Este tipo de figuras son útiles para elementos que requieren más espacio.}
\label{fig:diagrama-completo}
\end{figure*}

\lipsum[1-10] % Texto de ejemplo para llenar espacio