% Tufte-Latex/preamble.tex
% !TeX root = main.tex
% !TEX program = pdflatex


% Paquetes adicionales
\usepackage[spanish]{babel} % Idioma español
\usepackage{booktabs} % Tablas de calidad profesional
\usepackage{lipsum} % Texto de ejemplo
\usepackage{microtype} % Mejoras tipográficas
\UseMicrotypeSet[protrusion]{basicmath} % Ejemplo de configuración habitual
\microtypesetup{disable=footnote}
\usepackage{csquotes} % Recomendado por biblatex con babel/polyglossia
\usepackage{pdfpages}
\usepackage{tikz}
\usepackage{xurl}


% Configuración de márgenes y diseño
\geometry{
paperwidth=210mm,
paperheight=297mm,
left=42pt,
top=72pt,
textwidth=320pt,
marginparsep=20pt,
marginparwidth=180pt,
textheight=700pt, % Ajusta la altura del texto 650 original
footskip=40pt,
}


% Fuente sans-serif moderna
\usepackage{cmbright} % Fuente matematica moderna
% o 
%\usepackage{tgpagella}
%\usepackage{sfmath} 

% Configuraciion de fuente sanserif para todo el documento
\renewcommand*\familydefault{\sfdefault} % Cambia la familia global
\renewcommand{\rmdefault}{\sfdefault} % Cambia la familia de fuente a sans-serif
 

\usepackage{amsmath, amsfonts, mathtools, amsthm, amssymb} % Paquetes de matemáticas
%\usepackage{mathrsfs} % Para fuentes matemáticas adicionales

% Definimos el estilo myplain
\fancypagestyle{myplain}{
  \fancyhf{} % limpiar encabezados y pies
  \ifthenelse{\boolean{@tufte@twoside}}%
    {% two-side
      \fancyhead[LE]{\rightmark}
      \fancyhead[RO]{\MakeUppercase{\thechapter. ~\leftmark}}
      \fancyheadoffset[LE,RO]{-\marginparsep+20pt}
      \fancyfoot[LE,RO]{\thepage}
      \fancyfootoffset[LE,RO]{-\marginparsep+20pt}
    }%
    {% one-side
      \fancyhead[RE,RO]{\smallcaps{\newlinetospace{\plaintitle}}\quad\thepage}
    }%
}

% Enganchamos en \mainmatter
\let\oldmainmatter\mainmatter
\renewcommand{\mainmatter}{%
  \oldmainmatter
  \pagestyle{myplain}%
}

% % También para los capítulos en el mainmatter
% \let\oldchapter\chapter
% \renewcommand{\chapter}{%
%   \cleardoublepage
%   \pagestyle{myplain}
%   \oldchapter
% }


%Configuracion aspecto del titulo del capítulo
\usepackage{titlesec}
\usepackage{titletoc} % Carga el paquete helper
\usepackage{fmtcount} % Para formatear números en texto
\titleformat{\chapter}[display]
{\bfseries\Large\sf}
{\filleft
\textsl{\MakeUppercase{\chaptertitlename}} \expandafter\textsl\expandafter{\NUMBERstring{chapter}}
}
{1.5ex}
{\LARGE\rmfamily
\titlerule
\vspace*{1.1ex}%
\filright}
[\vspace*{1.1ex}%
\titlerule]



\titlecontents{chapter}
  [0pt] % Indent
  {\bfseries\Large\sf} % Format
  {\addvspace{0.5ex}{\MakeUppercase{\chaptertitlename}}~\thecontentslabel  \addvspace{0.5ex}} % Numbering format
  {\addvspace{0.5ex}} % Above code
  {\hfill\contentspage} % Page number format
  [\addvspace{0.5ex}] % Space after



\setcounter{secnumdepth}{1} % Profundidad de numeración de secciones (1 para capítulos, 2 para secciones, etc.)
\setcounter{tocdepth}{1} % Profundidad del índice (1 para capítulos, 2 para secciones, etc.)

\newcommand{\monthyear}{\ifcase\month\or January\or February\or March\or April\or May\or June\or July\or August\or September\or October\or November\or December\fi\space\number\year} % A command to print the current month and year


\addto\captionsspanish{ % Traducciones de términos comunes
  \renewcommand{\tablename}{Tabla} % Nombre de tablas y no cuadros
  \renewcommand{\listtablename}{Índice de tablas}
}

\usepackage{graphicx}
\DeclareGraphicsRule{.pspdftex}{pdf}{.pdftex}{}
\usepackage{grffile} % Permite nombres de archivos con espacios y caracteres especiales
\graphicspath{{./} {imagenes/}{~/OneDrive/TexPlantillas}} % Ruta de búsqueda para imágenes

% Metadatos para PDF
\usepackage{hyperref}
\hypersetup{
  pdftitle={Ejemplo completo de tufte-book},
  pdfauthor={John Doe},
  pdfsubject={LaTeX, tufte-book},
  pdfkeywords={latex, tufte, ejemplo, libro},
  bookmarksnumbered=true,
  bookmarksopen=true,
  pdfstartview={FitH},
  colorlinks=true,
  linkcolor=MidnightBlue,
  citecolor=ForestGreen,
  urlcolor=RubineRed,
}
