% !TeX root = main.tex
% !TEX program = pdflatex

\chapter{Introducción}

\section{Contextualización del Cáncer}
\label{sec:context-cancer}

El cáncer\footnote{El cáncer es una de las principales causas de muerte a nivel mundial, afectando a millones de personas cada año. Según la Organización Mundial de la Salud (OMS), se estima que en 2020 hubo aproximadamente 19.3 millones de nuevos casos y 10 millones de muertes por cáncer en todo el mundo. La detección temprana y el tratamiento adecuado son cruciales para mejorar las tasas de supervivencia.} se ha convertido en uno de los mayores desafíos de la salud pública alrededor del mundo, esto es debido a su elevada incidencia y su mortalidad. Es una enfermedad que afecta a millones de personas cada año, y representa una de las principales causas de muertes en muchos países, independientemente del nivel de desarrollo de estos mismos. No solo tiene una repercusión sanitaria, el cáncer tiene muchas repercusiones tanto económicas como sociales, ya que impacta de forma directa en la calidad de vida de los pacientes. El tratamiento del cáncer requiere un enfoque que debe combinar la prevención, el diagnóstico precoz, un tratamiento eficaz y un seguimiento continuo, a lo largo de este proceso deben de estar implicados profesionales de distintas disciplinas dentro del propio sistema sanitario. 

\subsection{Definición General del Cáncer}
\label{sec:def-cancer}

Se podría definir el cáncer como un grupo extenso de enfermedades, que se distinguen por la proliferación descontrolada e irregular de células que tienen la habilidad de infiltrarse y destruir tejidos cercanos, o incluso de extenderse hacia otros órganos a través de procesos metastásicos\cite{WHO2025}. La base biológica de esta enfermedad se encuentra en cambios genéticos y epigenéticos, ya sean heredados u adquiridos, que alteran procesos del ciclo celular, como la apoptosis o la multiplicación celular. Estas alteraciones genéticas propician que las células desarrollen conductas invasivas y se propaguen sin restricciones, generando de esta manera tumores malignos en cualquier tejido u órgano. 

\subsection{Impacto Mundial del Cáncer: Epidemiología y Tendencias Globales}
\label{sec:impact-cancer}

\paragraph[short]{Estadísticas Mundiales de Incidencia y Prevalencia}
%\label{sec:global-statistics}

A nivel global, el cáncer es una de las enfermedades más frecuentes y una de las principales causas de mortalidad. Según datos proporcionados por GLOBOCAN 2020\cite{globocan2020}, se estimó que en dicho año se produjeron aproximadamente 19,3 millones de nuevos casos de cáncer en el mundo. Los tipos de cáncer más diagnosticados a nivel global son el cáncer de mama, pulmón, colorrectal, próstata y gástrico. La incidencia global continúa aumentando, principalmente debido al envejecimiento poblacional y al incremento en la prevalencia de factores de riesgo asociados al estilo de vida.

\paragraph[short]{Tendencias en la Incidencia del Cáncer}
%\label{sec:global-trends}
Las tendencias en la incidencia del cáncer muestran un aumento constante en la mayoría de las regiones del mundo, especialmente en países en desarrollo. Este incremento se atribuye a varios factores, incluyendo el envejecimiento de la población, la urbanización y la adopción de estilos de vida poco saludables, como el tabaquismo, el sedentarismo y dietas poco saludables. Además, se ha observado un aumento en la detección temprana y el diagnóstico de cáncer, lo que también contribuye al aumento aparente de casos\cite{Zhou2024}.
\subsection{Supervivencia y Mortalidad por Cáncer}
\label{sec:survival-mortality}


\paragraph[short]{Supervivencia y mortalidad según el tipo de cáncer}
% \label{sec:survival-mortality} 

Las tasas de supervivencia en cáncer varían considerablemente según el tipo específico y el estadio en el que se realiza el diagnóstico. Por ejemplo, el cáncer de mama presenta tasas de supervivencia a 5 años superiores al 80\% en países desarrollados gracias a la detección precoz y mejores tratamientos disponibles. En cambio, el cáncer de pulmón muestra tasas de supervivencia significativamente menores, inferiores al 20\%, principalmente debido a su diagnóstico tardío y la agresividad propia del tumor\cite{allemani2018global}. Otros tipos de cáncer con peor pronóstico incluyen el cáncer de páncreas, con una supervivencia a cinco años que raramente supera el 10\%, dada la dificultad en su detección temprana. 

\paragraph[short]{Tasas globales de mortalidad por cáncer}
% \label{sec:global-mortality} 

A nivel mundial, el cáncer constituye la segunda causa de muerte más común después de las enfermedades cardiovasculares, siendo responsable de aproximadamente uno de cada seis fallecimientos anuales en el mundo. En 2020, aproximadamente 10 millones de personas fallecieron por causa directa del cáncer. El cáncer de pulmón es el más letal, seguido del cáncer colorrectal y hepático, siendo estos tres tipos responsables de casi la mitad del total de muertes relacionadas con el cáncer\cite{Sung2021}. 

\subsection{Estadísticas del Cáncer en España}
\label{sec:stats-cancer-spain}

\paragraph{Casos nuevos diagnosticados anualmente}

En España, el cáncer continúa siendo uno de los problemas de salud pública más importantes, registrándose cada año aproximadamente 280.100 nuevos casos según el informe de la Sociedad Española de Oncología Médica (SEOM)  \cite{SEOM2025}. Esta cifra se prevé que continúe aumentando en las próximas décadas debido al envejecimiento de la población española y el impacto acumulativo de factores ambientales y hábitos de vida. 

\paragraph{Tipos más frecuentes de cáncer en España} 

En la población española, se han detectado los siguientes cánceres como los más frecuentes: cáncer colorectal, cáncer de próstata, cáncer de pulmón, cáncer de mama y cáncer de vejiga urinaria. En el caso de los hombres, dominan los cánceres de próstata, pulmón y colon, mientras que, en el caso de las mujeres, el cáncer más frecuente es el cáncer de mama, seguido por el cáncer de colon y el cáncer de pulmón, estos datos están condicionados en parte por los hábitos de vida establecidos y por las características demográficas particulares del país como indica la Sociedad Española de Oncología Médica (SEOM). 

\subsection{Factores Influyentes en la Incidencia del Cáncer}
\label{sec:influencing-factors} 

El cáncer es una enfermedad multifactorial, influenciada por factores demográficos, genéticos y ambientales, siendo fundamental identificar estos factores para establecer estrategias preventivas eficaces. 

\paragraph{Factores demográficos} 

Entre los factores demográficos más relevantes está la edad, ya que la mayoría de los tipos de cáncer tienen una incidencia notablemente mayor en personas mayores de 60 años, debido a la acumulación de mutaciones genéticas a lo largo de la vida. Asimismo, el sexo condiciona considerablemente la incidencia de ciertos tipos específicos de cáncer, como el cáncer de próstata, que es exclusivamente masculino, o el cáncer de mama, que presenta una mayor incidencia en mujeres\cite{cancer_risk_factors}. Por último, la herencia genética también representa un factor importante para algunos tipos de cáncer, como ocurre con el cáncer de mama y ovario en pacientes con mutaciones en los genes BRCA1 y BRCA2. 

\paragraph{Factores de riesgo asociados al estilo de vida} 

Diversos estudios han demostrado que los factores asociados al estilo de vida juegan un papel decisivo en la incidencia de cáncer. El tabaquismo constituye el principal factor de riesgo evitable para numerosos tipos de cáncer, especialmente cáncer de pulmón, cabeza y cuello, y vías urinarias\cite{restrepo1988relacion}. Otros factores relacionados con hábitos de vida son el consumo excesivo de alcohol, una dieta rica en grasas saturadas y pobre en fibra, la obesidad, el sedentarismo y la exposición prolongada al sol sin protección, factores todos ellos que incrementan significativamente el riesgo de desarrollar diferentes tipos de cáncer\cite{PAPPS2024}. Por ello, la prevención primaria, mediante cambios en el estilo de vida, representa una estrategia clave en la reducción del riesgo global y en la disminución de la incidencia de cáncer.

\subsection{Importancia de la Detección Temprana y el Tratamiento}
\label{sec:importance-early-detection}
La detección temprana del cáncer es crucial para mejorar las tasas de supervivencia y la calidad de vida de los pacientes. La identificación precoz de la enfermedad permite iniciar tratamientos más efectivos y menos invasivos, lo que puede llevar a una mayor probabilidad de curación o control de la enfermedad. Las estrategias de detección temprana incluyen programas de cribado poblacional, como mamografías para el cáncer de mama, colonoscopias para el cáncer colorrectal y pruebas de Papanicolaou para el cáncer cervical. Estos programas han demostrado reducir la mortalidad por cáncer en poblaciones específicas al facilitar diagnósticos en etapas más tratables\cite{usptf2023cancer}.
\subsection{Avances en la Investigación y Tratamiento del Cáncer}
\label{sec:advances-cancer-research}          
La investigación en cáncer ha avanzado significativamente en las últimas décadas, lo que ha llevado al desarrollo de nuevas terapias y enfoques de tratamiento. Los avances en la biología molecular y la genética han permitido identificar biomarcadores específicos que pueden predecir la respuesta a ciertos tratamientos, lo que ha llevado a la personalización de la terapia oncológica\cite{Hanahan2022}. Además, el desarrollo de terapias dirigidas y la inmunoterapia han revolucionado el tratamiento del cáncer\cite{Schreiber2021}, ofreciendo nuevas esperanzas para pacientes con tumores previamente considerados incurables. Estos avances no solo mejoran las tasas de supervivencia, sino que también reducen los efectos secundarios asociados con tratamientos más agresivos\cite{Mukherjee2010}.

\section{La Planificación Quirúrgica en Oncología}
\label{sec:oncology-surgical-planning}

\subsection{Concepto General de la Planificación Quirúrgica}
\label{sec:general-concept-surgical-planning}
La planificación quirúrgica se refiere al conjunto de procesos y estrategias empleadas por los equipos de clínicos para diseñar, optimizar y anticipar los pasos necesarios para llevar a cabo una intervención quirúrgica. Este proceso incluye no solo la identificación precisa de la patología a tratar, sino también la determinación detallada de las estructuras anatómicas implicadas, la evaluación de posibles riesgos y complicaciones, y la selección del abordaje quirúrgico más adecuado según el contexto clínico específico de cada paciente\cite{Gillies2020}. 

El objetivo principal de la planificación quirúrgica es maximizar la eficacia y seguridad del procedimiento, buscando eliminar o controlar la patología tumoral con la menor afectación posible de estructuras vitales adyacentes, preservando así al máximo la función de los órgano y tejidos sanos. Esto se traduce directamente en una mejora de la calidad de vida del paciente tras la cirugía, reduciendo significativamente los riesgos postoperatorios, la duración del periodo de recuperación y mejorando las tasas globales de supervivencia\cite{Birkmeyer2012}. 

Una planificación quirúrgica precisa y rigurosa resulta particularmente importante en cirugía oncológica, donde la exactitud en la identificación y delimitación tumoral puede ser determinante para conseguir resecciones completas, con márgenes libres adecuados, reduciendo el riesgo de recurrencias o metástasis futura\cite{NCCN2023}.


